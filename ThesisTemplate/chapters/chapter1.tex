\chapter{Introduction\label{cha:chapter1}}

In this world of technological and innovation development, we have seen a continuous
evolution in types of businesses. Though the objective of each type of firm is same i.e. to generate
revenue by creating value for customers, there are differences in their approach to innovate and
maintain sustainability. Two types of businesses which have been the focus of attention to
entrepreneurs and researchers for many years are spin-offs and start-ups. Both businesses have contributed to
improve nation’s economies and employment rates. However, there are differences in their
innovation capabilities and their establishment processes.


\section{Motivation\label{sec:moti}}
In entrepreneurial literature, there have been several studies for understanding the taxonomy and formation of spin-off firms and start-ups.
Researchers have been trying to narrow the gap between theoretical understanding of processes adopted by spin-offs and start-ups and their related practical implications. Both businesses have been studied to understand their topologies, reasons of success and failures \cite{fastcompany}, and their affects on
the economical growth\cite{economical_growth}. There are also numerous comparative studies between spin-offs and
start-ups based on technological development \cite{comparative_studies}, resource inheritance and exploitation \cite{resource_inheritance}, and
survival and growth rate \cite{whose_child}. However, there has been little research on finding the major factors which
affect the innovation processes of start-ups and spin-offs.

\section{Objective\label{sec:objective}}
The purpose of this paper is to investigate major factors which affect the innovation methods of
spin-offs and start-ups. It has addressed this problem by reviewing existing literature and using
exploratory interviews with founders of spin-offs and start-ups. This study will guide the future research in this domain and facilitates business owners to
develop strategies for their business’s success.


\section{Outline\label{sec:outline}}

The remainder of this paper is summarized as follows:
\\
In \textbf{Section \ref{cha:chapter2}}, prior research will be discussed
in order to lay foundations of the problem in hand.
\\
\\
Based on the theoretical discussion,\textbf{Section \ref{cha:chapter3}} will describe theoretical framework which will help to identify major factors for comparison and derive hypotheses. 
\\
\\
In \textbf{Section \ref{cha:chapter4}}, methodology of collecting data for testing hypotheses will be
discussed
\\
\\
\textbf{Section \ref{cha:chapter5}} will analyze the data and derive meaningful results.
\\
\\
Final \textbf{Section \ref{cha:chapter6}} concludes the research, discusses its potential benefits and identifies the opportunities for future
work.
