\thispagestyle{empty}
\vspace*{0.2cm}

\begin{center}
    \textbf{Zusammenfassung}
\end{center}

\vspace*{0.2cm}

\noindent 
In den letzten Jahren wurde unser Leben durch revolution�re Technologien drastisch verbessert
Innovationen Dies wurde durch Visionen von Unternehmern und F�hrungskr�ften erm�glicht
Unternehmen wie Spin-offs und Start-ups. Sowohl Spin-offs als auch Start-ups wurden weithin angenommen
Unternehmerische Welt und haben sich als sehr erfolgreich in Bezug auf die Wettbewerbsf�higkeit,
Innovationskraft, Wachstum und wirtschaftliche Entwicklung. Allerdings gibt es noch eine L�cke zum Verst�ndnis der
Gro�e Unterschiede in der Innovationsf�higkeit von Spin-offs und Start-ups. Diese Forschung ist fokussiert
Auf die wichtigsten Faktoren, die die Innovationsleistung von Start-ups beeinflussen
Und Spin-offs Literarische theorien i.e. Ressource-basierte Theorie, Humankapitaltheorie, Sozialkapitaltheorie und Motivationstheorie verwendet worden war
Um solche Faktoren zu finden und entwickelte Hypothesen, die mit Literaturrezensionen und qualitativen explorativen Frageb�gen getestet wurden.
Diese Studie wird dazu beitragen, den Weg f�r die zuk�nftige theoretische und experimentelle Forschung auf zu weben
Vergleich von Start-ups mit Spin-offs in Bezug auf Innovationsentwicklungen und helfen Unternehmern, ihre Innovationen zu planen
Strategien effizient.
\\
\medskip


\medskip
\noindent \textit{Schl�sselw�rter}: Inbetriebnahme, Spin-offs, Innovationsf�higkeit, Innovationsleistung, \\
Ressourcenbasierte Theorie, Humankapitaltheorie, Sozialkapitaltheorie, Motivationstheorie