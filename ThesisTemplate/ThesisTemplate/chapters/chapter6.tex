\chapter{Conclusion}
Main purpose of this research was to investigate the differences between innovation capabilities of
spin-offs and start-ups. In order to achieve goal of this study, hypotheses were developed regarding
major factors affecting innovation capabilities of spin-offs and start-ups. Four major factors were
chosen using literary theories which are resource-based theory, human capital theory, social capital
theory and motivational theory. In order to test hypotheses, qualitative methodology is used.
Questionnaire was being sent to three respondents which hold administrative and managerial positions in the
spin-offs and start-ups. For secondary research, books and research papers were consulted. The
results showed that transfer of resources from parent companies, previous industry experience,
management skills and networking relations help spin-offs to perform better than individual start-ups.However, spin-offs which are developed as a result of adverse affects and inefficient driving
forces in parent companies, cause lack of motivation in employees which badly affects
performance.

\section{Potential Benefits}
The conclusion of this research will help future researchers in comparative studies of spin-offs and
start-ups. It will help to further investigate the factors affecting innovation capabilities of firms.
Moreover, it can be useful to business owners for understanding the reasons for success and failures
of firms ,develop strategies for innovation development and evaluate main
challenges on road of success.
\section{Future Work}
This research can be used for future investigation of important factors impacting
the development of innovation activities of firms. More in-depth investigation and interviews can be conducted
to make the results of this study more reliable. For future work, it would be very interesting to investigate
factors impacting innovation of spin-offs and start-ups in different countries and in different development stages
of firms.
