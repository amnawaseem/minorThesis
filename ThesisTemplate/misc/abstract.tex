\thispagestyle{empty}
\vspace*{1.0cm}

\begin{center}
    \textbf{Abstract}
\end{center}

\vspace*{0.5cm}

\noindent
In recent years, our lives have been drastically improved by revolutionary technologies and
innovations. This has been made possible by visions of entrepreneurs and leaders working in
businesses like spin-offs and start-ups. Both spin-offs and start-ups have been widely adopted in
entrepreneurial world and have been proven to be highly successful in terms of competitiveness,
innovativeness, growth, and economical development. However, there is still a gap in understanding the
major differences in innovation capabilities of spin-offs and start-ups. This research is focused
on finding the major factors which affect the innovation performance of start-ups
and spin-offs. Literary theories i.e. resource-based theory, human capital theory, social capital theory and motivational theory had been used in this paper
to find such factors and developed hypotheses which were tested using literature reviews and qualitative exploratory questionnaires. 
This study will help weave the path for future theoretical and experimental research on
comparing start-ups with spin-offs regarding innovation developments and help business owners to plan their innovative
strategies efficiently.
\\
\medskip

%\noindent \textit{JEL classification}: XXX, YYY.

\medskip
\noindent \textit{Keywords}: Startup, Spin-offs, Innovation Capabilities, Innovation Performance, Resource-based Theory, Human Capital Theory, Social Capital Theory, Motivational Theory